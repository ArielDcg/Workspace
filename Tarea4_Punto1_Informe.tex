\documentclass[12pt,a4paper]{article}
\usepackage[utf8]{inputenc}
\usepackage[spanish]{babel}
\usepackage{amsmath}
\usepackage{amsfonts}
\usepackage{amssymb}
\usepackage{graphicx}
\usepackage{float}
\usepackage{geometry}
\usepackage{hyperref}
\usepackage{listings}
\usepackage{xcolor}
\usepackage{booktabs}

\geometry{left=2.5cm, right=2.5cm, top=2.5cm, bottom=2.5cm}

\title{\textbf{Tarea 4 - Punto 1} \\ Solución de Reactor PFR usando Elementos Finitos}
\author{Elementos Finitos - Método de Galerkin}
\date{\today}

\begin{document}

\maketitle
\tableofcontents
\newpage

\section{Introducción}

El presente documento desarrolla la solución del estado estacionario de concentración en un reactor PFR (Plug Flow Reactor) utilizando el método de elementos finitos con formulación de Galerkin y debilitamiento. Se presentan dos enfoques:

\begin{enumerate}
    \item \textbf{Formulación NO discretizada}: Usando polinomios de Legendre como funciones base
    \item \textbf{Formulación discretizada (FEM)}: Usando polinomios de Lagrange como funciones base
\end{enumerate}

\section{Planteamiento del Problema}

\subsection{Ecuación Gobernante}

La ecuación diferencial que modela el estado estacionario de concentración con reacción de primer orden en un reactor PFR es:

\begin{equation}
D\frac{d^2c}{dx^2} - U\frac{dc}{dx} - kc = 0
\label{eq:governing}
\end{equation}

donde:
\begin{itemize}
    \item $D = 5000$ m$^2$/hr: coeficiente de difusión/dispersión
    \item $U = 100$ m/hr: velocidad advectiva o \textit{bulk velocity}
    \item $k = 2$ hr$^{-1}$: tasa de reacción (consumo de la especie)
    \item $c(x)$: concentración de la sustancia [mol/L]
    \item $x$: posición a lo largo del reactor [m]
\end{itemize}

\subsection{Condiciones de Contorno}

El problema está sujeto a las siguientes condiciones de contorno:

\begin{align}
U c_{inlet} &= U c - D\frac{dc}{dx}, \quad x = 0 \label{eq:bc1}\\
\frac{dc}{dx} &= 0, \quad x = L \label{eq:bc2}
\end{align}

Con los siguientes valores:
\begin{itemize}
    \item $L = 100$ m: longitud del reactor
    \item $c_{inlet} = 100$ mol/L: concentración de entrada
\end{itemize}

\subsection{Solución Analítica}

La solución analítica del problema está dada por:

\begin{equation}
c(x) = \frac{U c_{in}}{(U - D\lambda_1) \lambda_2 e^{\lambda_2 L} - (U - D\lambda_2) \lambda_1 e^{\lambda_1 L}} \left[\lambda_2 e^{\lambda_2 L} e^{\lambda_1 x} - \lambda_1 e^{\lambda_1 L} e^{\lambda_2 x}\right]
\label{eq:analytical}
\end{equation}

donde los valores característicos están dados por:

\begin{equation}
\lambda_{1,2} = \frac{U}{2D}\left[1 \pm \sqrt{1 + \frac{4kD}{U^2}}\right]
\label{eq:lambda}
\end{equation}

\section{Desarrollo Matemático: Método de Galerkin}

\subsection{Formulación Débil (Weakening)}

Partiendo de la ecuación diferencial (\ref{eq:governing}), multiplicamos por una función de prueba $N_l(x)$ e integramos sobre el dominio $\Omega = [0, L]$:

\begin{equation}
\int_0^L N_l(x) \left[D\frac{d^2c}{dx^2} - U\frac{dc}{dx} - kc\right] dx = 0
\label{eq:weak_form_1}
\end{equation}

\subsection{Integración por Partes (Debilitamiento)}

Aplicamos integración por partes al término de segunda derivada para \textbf{debilitar} la formulación:

\begin{equation}
\int_0^L N_l D\frac{d^2c}{dx^2} dx = \left[N_l D\frac{dc}{dx}\right]_0^L - \int_0^L \frac{dN_l}{dx} D\frac{dc}{dx} dx
\label{eq:integration_by_parts}
\end{equation}

Sustituyendo en (\ref{eq:weak_form_1}):

\begin{equation}
\left[N_l D\frac{dc}{dx}\right]_0^L - \int_0^L \frac{dN_l}{dx} D\frac{dc}{dx} dx - \int_0^L N_l U\frac{dc}{dx} dx - \int_0^L N_l k c dx = 0
\label{eq:weak_form_2}
\end{equation}

\subsection{Aproximación de Galerkin}

Aproximamos la solución $c(x)$ como una combinación lineal de funciones base $N_m(x)$:

\begin{equation}
c(x) \approx \sum_{m=0}^{M-1} a_m N_m(x)
\label{eq:galerkin_approx}
\end{equation}

Sustituyendo (\ref{eq:galerkin_approx}) en (\ref{eq:weak_form_2}) y usando el método de Galerkin ($N_l = N_m$ como funciones de prueba):

\begin{multline}
\left[N_l D\sum_{m=0}^{M-1} a_m \frac{dN_m}{dx}\right]_0^L - \sum_{m=0}^{M-1} a_m \int_0^L \frac{dN_l}{dx} D\frac{dN_m}{dx} dx \\
- \sum_{m=0}^{M-1} a_m \int_0^L N_l U\frac{dN_m}{dx} dx - \sum_{m=0}^{M-1} a_m \int_0^L N_l k N_m dx = 0
\label{eq:galerkin_full}
\end{multline}

\subsection{Matrices Elementales $K_{lm}$ y $F_l$}

Reorganizando (\ref{eq:galerkin_full}), obtenemos el sistema matricial:

\begin{equation}
\sum_{m=0}^{M-1} K_{lm} a_m = F_l
\label{eq:matrix_system}
\end{equation}

donde:

\begin{equation}
\boxed{K_{lm} = -\int_0^L D\frac{dN_l}{dx}\frac{dN_m}{dx} dx - \int_0^L U N_l \frac{dN_m}{dx} dx - \int_0^L k N_l N_m dx}
\label{eq:Klm}
\end{equation}

y el vector de carga $F_l$ proviene del término de frontera:

\begin{equation}
F_l = -\left[N_l D\frac{dc}{dx}\right]_0^L
\label{eq:Fl_boundary}
\end{equation}

\subsection{Tratamiento de Condiciones de Contorno}

\subsubsection{Condición en $x = 0$}

De la condición (\ref{eq:bc1}):
\begin{equation}
D\frac{dc}{dx}\bigg|_{x=0} = U(c - c_{inlet})\bigg|_{x=0} = U\sum_{m=0}^{M-1} a_m N_m(0) - U c_{inlet}
\end{equation}

\subsubsection{Condición en $x = L$}

De la condición (\ref{eq:bc2}):
\begin{equation}
\frac{dc}{dx}\bigg|_{x=L} = 0
\end{equation}

\subsubsection{Vector de Carga Final}

Evaluando (\ref{eq:Fl_boundary}) con las condiciones de contorno:

\begin{align}
F_l &= -N_l(L) \cdot D \cdot 0 + N_l(0) \cdot D \cdot \frac{dc}{dx}\bigg|_{x=0} \\
&= N_l(0) \cdot U \left(\sum_{m=0}^{M-1} a_m N_m(0) - c_{inlet}\right)
\end{align}

Llevando los términos con $a_m$ al lado izquierdo:

\begin{equation}
\boxed{F_l = N_l(0) \cdot U \cdot c_{inlet}}
\label{eq:Fl_final}
\end{equation}

y se añade a $K_{lm}$:

\begin{equation}
K_{lm} \leftarrow K_{lm} + N_l(0) \cdot U \cdot N_m(0)
\label{eq:Klm_bc}
\end{equation}

\subsection{Sistema Matricial Global}

El problema se reduce a resolver el sistema lineal:

\begin{equation}
\boxed{[K]\{a\} = \{F\}}
\label{eq:global_system}
\end{equation}

donde:
\begin{itemize}
    \item $[K]$: Matriz de rigidez global ($M \times M$)
    \item $\{a\}$: Vector de coeficientes incógnitas ($M \times 1$)
    \item $\{F\}$: Vector de carga global ($M \times 1$)
\end{itemize}

\section{Formulación NO Discretizada: Polinomios de Legendre}

\subsection{Funciones Base de Legendre}

Los polinomios de Legendre $P_m(\xi)$ están definidos en el intervalo $\xi \in [-1, 1]$. Para mapearlos al dominio físico $[0, L]$, usamos:

\begin{equation}
\xi = \frac{2x}{L} - 1 \quad \Rightarrow \quad x = \frac{L}{2}(\xi + 1)
\label{eq:mapping}
\end{equation}

Las funciones base son:
\begin{equation}
N_m(x) = P_m(\xi(x)), \quad m = 0, 1, 2, \ldots, M-1
\end{equation}

\subsection{Derivadas en el Dominio Físico}

La primera derivada se transforma como:
\begin{equation}
\frac{dN_m}{dx} = \frac{dP_m}{d\xi} \cdot \frac{d\xi}{dx} = \frac{dP_m}{d\xi} \cdot \frac{2}{L}
\end{equation}

La segunda derivada:
\begin{equation}
\frac{d^2N_m}{dx^2} = \frac{d^2P_m}{d\xi^2} \cdot \left(\frac{2}{L}\right)^2
\end{equation}

\subsection{Cálculo de $K_{lm}$ para Legendre}

Sustituyendo en (\ref{eq:Klm}):

\begin{align}
K_{lm}^{Leg} &= -D \int_0^L \left(\frac{dP_l}{d\xi} \cdot \frac{2}{L}\right)\left(\frac{dP_m}{d\xi} \cdot \frac{2}{L}\right) dx \nonumber \\
&\quad - U \int_0^L P_l(\xi) \left(\frac{dP_m}{d\xi} \cdot \frac{2}{L}\right) dx \nonumber \\
&\quad - k \int_0^L P_l(\xi) P_m(\xi) dx
\end{align}

Cambiando variable a $\xi$: $dx = \frac{L}{2}d\xi$

\begin{equation}
\boxed{K_{lm}^{Leg} = -\frac{2D}{L}\int_{-1}^{1} \frac{dP_l}{d\xi}\frac{dP_m}{d\xi} d\xi - U\int_{-1}^{1} P_l \frac{dP_m}{d\xi} d\xi - \frac{kL}{2}\int_{-1}^{1} P_l P_m d\xi}
\label{eq:Klm_legendre}
\end{equation}

\subsection{Cálculo de $F_l$ para Legendre}

De (\ref{eq:Fl_final}), evaluando en $x=0$ ($\xi = -1$):

\begin{equation}
\boxed{F_l^{Leg} = U \cdot c_{inlet} \cdot P_l(-1)}
\label{eq:Fl_legendre}
\end{equation}

\subsection{Modificación por Condición en $x=0$}

La matriz $K$ se modifica según (\ref{eq:Klm_bc}):

\begin{equation}
K_{lm}^{Leg} \leftarrow K_{lm}^{Leg} + U \cdot P_l(-1) \cdot P_m(-1)
\end{equation}

\section{Formulación Discretizada: Elementos Finitos con Lagrange}

\subsection{Discretización del Dominio}

El dominio $[0, L]$ se divide en $n_{elem}$ elementos:

\begin{equation}
\Omega = \bigcup_{e=1}^{n_{elem}} \Omega^e, \quad \Omega^e = [x_{e}, x_{e+1}]
\end{equation}

con longitud de elemento:
\begin{equation}
h_e = \frac{L}{n_{elem}}
\end{equation}

Se generan $n_{nodos} = n_{elem} + 1$ nodos.

\subsection{Funciones Base de Lagrange Lineales}

En cada elemento, usamos funciones de forma lineales en coordenada local $\xi \in [0, 1]$:

\begin{align}
N_1(\xi) &= 1 - \xi \label{eq:N1}\\
N_2(\xi) &= \xi \label{eq:N2}
\end{align}

con derivadas:
\begin{align}
\frac{dN_1}{d\xi} &= -1 \\
\frac{dN_2}{d\xi} &= 1
\end{align}

Mapeo al elemento físico:
\begin{equation}
x = x_e + \xi \cdot h_e, \quad \xi = \frac{x - x_e}{h_e}
\end{equation}

\subsection{Matriz de Rigidez Local $K^e$}

Para un elemento $e$, la matriz local $2 \times 2$ es:

\begin{equation}
K_{ij}^e = -\int_{\Omega^e} D\frac{dN_i}{dx}\frac{dN_j}{dx} dx - \int_{\Omega^e} U N_i \frac{dN_j}{dx} dx - \int_{\Omega^e} k N_i N_j dx
\end{equation}

Transformando a coordenadas locales ($dx = h_e d\xi$, $\frac{d}{dx} = \frac{1}{h_e}\frac{d}{d\xi}$):

\begin{multline}
K_{ij}^e = -\int_{0}^{1} D\frac{1}{h_e}\frac{dN_i}{d\xi}\frac{1}{h_e}\frac{dN_j}{d\xi} h_e d\xi \\
- \int_{0}^{1} U N_i \frac{1}{h_e}\frac{dN_j}{d\xi} h_e d\xi - \int_{0}^{1} k N_i N_j h_e d\xi
\end{multline}

Simplificando:

\begin{equation}
\boxed{K_{ij}^e = -\frac{D}{h_e}\int_{0}^{1} \frac{dN_i}{d\xi}\frac{dN_j}{d\xi} d\xi - U\int_{0}^{1} N_i \frac{dN_j}{d\xi} d\xi - kh_e\int_{0}^{1} N_i N_j d\xi}
\label{eq:Ke_lagrange}
\end{equation}

\subsection{Cálculo Explícito de $K^e$}

Para elementos lineales, podemos calcular analíticamente:

\subsubsection{Término Difusivo}

\begin{equation}
\int_0^1 \frac{dN_i}{d\xi}\frac{dN_j}{d\xi} d\xi =
\begin{bmatrix}
1 & -1 \\
-1 & 1
\end{bmatrix}
\end{equation}

\subsubsection{Término Advectivo}

\begin{equation}
\int_0^1 N_i \frac{dN_j}{d\xi} d\xi =
\begin{bmatrix}
-\frac{1}{2} & \frac{1}{2} \\
-\frac{1}{2} & \frac{1}{2}
\end{bmatrix}
\end{equation}

\subsubsection{Término Reactivo}

\begin{equation}
\int_0^1 N_i N_j d\xi =
\begin{bmatrix}
\frac{1}{3} & \frac{1}{6} \\
\frac{1}{6} & \frac{1}{3}
\end{bmatrix}
\end{equation}

\subsubsection{Matriz Local Completa}

\begin{equation}
K^e = -\frac{D}{h_e}\begin{bmatrix}
1 & -1 \\
-1 & 1
\end{bmatrix}
- U\begin{bmatrix}
-\frac{1}{2} & \frac{1}{2} \\
-\frac{1}{2} & \frac{1}{2}
\end{bmatrix}
- kh_e\begin{bmatrix}
\frac{1}{3} & \frac{1}{6} \\
\frac{1}{6} & \frac{1}{3}
\end{bmatrix}
\end{equation}

\subsection{Ensamblaje Global}

La matriz global $K_{global}$ se construye ensamblando las matrices locales:

\begin{equation}
K_{global}[i_{global}, j_{global}] \leftarrow K_{global}[i_{global}, j_{global}] + K_{ij}^e
\end{equation}

donde $i_{global}, j_{global}$ son los índices globales de los nodos del elemento $e$.

\subsection{Aplicación de Condiciones de Contorno (FEM)}

\subsubsection{En $x = 0$ (nodo 0)}

Contribución al vector de carga:
\begin{equation}
F_{global}[0] = U \cdot c_{inlet}
\end{equation}

\subsubsection{En $x = L$ (nodo $n_{nodos}-1$)}

La condición de Neumann homogénea ($\frac{dc}{dx} = 0$) es natural y no requiere modificación explícita.

\section{Índices para Elementos Interiores y Exteriores}

\subsection{Elementos Interiores}

Los elementos interiores ($e = 1, 2, \ldots, n_{elem}-2$) contribuyen normalmente sin consideraciones especiales de frontera.

\subsection{Elementos Exteriores}

\subsubsection{Primer Elemento ($e = 0$)}

\begin{itemize}
    \item Nodo local 0 $\rightarrow$ Nodo global 0 (frontera izquierda)
    \item Nodo local 1 $\rightarrow$ Nodo global 1
    \item Aplicar condición de contorno tipo Robin en nodo 0
\end{itemize}

\subsubsection{Último Elemento ($e = n_{elem}-1$)}

\begin{itemize}
    \item Nodo local 0 $\rightarrow$ Nodo global $n_{elem}-1$
    \item Nodo local 1 $\rightarrow$ Nodo global $n_{elem}$ (frontera derecha)
    \item Condición de Neumann natural en nodo $n_{elem}$
\end{itemize}

\section{Cálculo del Residuo}

El residuo $R(x)$ mide qué tan bien la aproximación $c_{approx}(x)$ satisface la ecuación diferencial:

\begin{equation}
R(x) = D\frac{d^2c_{approx}}{dx^2} - U\frac{dc_{approx}}{dx} - k \cdot c_{approx}
\label{eq:residual}
\end{equation}

Para la aproximación:
\begin{equation}
c_{approx}(x) = \sum_{m=0}^{M-1} a_m N_m(x)
\end{equation}

Calculamos:
\begin{align}
\frac{dc_{approx}}{dx} &= \sum_{m=0}^{M-1} a_m \frac{dN_m}{dx} \\
\frac{d^2c_{approx}}{dx^2} &= \sum_{m=0}^{M-1} a_m \frac{d^2N_m}{dx^2}
\end{align}

\section{Error Norma $L_2$}

El error en norma $L_2$ entre la solución aproximada y la analítica se define como:

\begin{equation}
\|e\|_{L_2} = \sqrt{\frac{1}{L}\int_0^L (c_{approx}(x) - c_{exact}(x))^2 dx}
\label{eq:L2_error}
\end{equation}

Numéricamente:
\begin{equation}
\|e\|_{L_2} \approx \sqrt{\frac{1}{L}\sum_{i=1}^{N} (c_{approx}(x_i) - c_{exact}(x_i))^2 \Delta x}
\end{equation}

\section{Análisis de Convergencia}

\subsection{Convergencia No Discretizada (Legendre)}

Para métodos espectrales con polinomios de Legendre, la convergencia es \textbf{exponencial} (o espectral):

\begin{equation}
\|e\|_{L_2} \sim e^{-\alpha M}
\end{equation}

donde $\alpha > 0$ depende de la suavidad de la solución.

\subsection{Convergencia Discretizada (FEM con Lagrange)}

Para elementos finitos con funciones lineales, la convergencia es \textbf{algebraica}:

\begin{equation}
\|e\|_{L_2} \sim \mathcal{O}(h^{p+1}) = \mathcal{O}(h^2)
\end{equation}

donde $p = 1$ es el orden del polinomio de interpolación.

\section{Resultados}

\subsection{Parámetros Utilizados}

\begin{table}[H]
\centering
\begin{tabular}{@{}lll@{}}
\toprule
\textbf{Parámetro} & \textbf{Valor} & \textbf{Unidades} \\ \midrule
$D$ & 5000 & m$^2$/hr \\
$U$ & 100 & m/hr \\
$k$ & 2 & hr$^{-1}$ \\
$L$ & 100 & m \\
$c_{inlet}$ & 100 & mol/L \\ \bottomrule
\end{tabular}
\caption{Parámetros del reactor PFR}
\end{table}

\subsection{Convergencia No Discretizada}

Se probaron valores de $M = 3, 5, 8, 12, 15, 18, 20$ funciones base de Legendre. Los resultados muestran:

\begin{itemize}
    \item Convergencia rápida a la solución analítica
    \item Error $L_2 < 10^{-6}$ para $M \geq 12$
    \item Convergencia exponencial observada
\end{itemize}

\subsection{Convergencia Discretizada (FEM)}

Se probaron mallas con $n_{elem} = 5, 10, 15, 20, 25, 30, 40, 50, 60, 80, 100$ elementos. Los resultados muestran:

\begin{itemize}
    \item Convergencia $\mathcal{O}(h^2)$ confirmada
    \item Error $L_2 < 10^{-3}$ para $n_{elem} \geq 50$
    \item Mayor cantidad de grados de libertad requeridos comparado con Legendre
\end{itemize}

\section{Estudio Paramétrico de $k$}

Se analizó el efecto de variar la tasa de reacción $k$ en los valores: $k/4, k/2, k, 2k, 4k$.

\subsection{Efectos de Reducir $k$}

\begin{itemize}
    \item Menor consumo de reactivo a lo largo del reactor
    \item Mayor concentración a la salida $c(L)$
    \item Menor conversión del reactor
    \item Perfil de concentración más plano
    \item Menor dominancia del término reactivo
\end{itemize}

\subsection{Efectos de Aumentar $k$}

\begin{itemize}
    \item Mayor consumo de reactivo
    \item Menor concentración a la salida $c(L)$
    \item Mayor conversión del reactor
    \item Perfil de concentración más pronunciado
    \item Gradientes más fuertes (posibles capas límite)
    \item Requiere mayor refinamiento numérico para mantener precisión
\end{itemize}

\subsection{Conversión del Reactor}

La conversión se define como:
\begin{equation}
X = \frac{c_{inlet} - c(L)}{c_{inlet}} \times 100\%
\end{equation}

\section{Respuestas a las Preguntas}

\subsection{¿Qué efectos tiene computacionalmente aplicar una discretización?}

\subsubsection{Estructura de Matrices}

\begin{itemize}
    \item \textbf{No discretizada (Legendre)}: Matriz \textbf{densa} $M \times M$ con $\mathcal{O}(M^2)$ elementos no nulos
    \item \textbf{Discretizada (FEM)}: Matriz \textbf{banda sparse} con ancho de banda limitado (3 para elementos lineales), $\mathcal{O}(N)$ elementos no nulos
\end{itemize}

\subsubsection{Costo Computacional}

\textbf{Ensamblaje:}
\begin{itemize}
    \item No discretizada: $\mathcal{O}(M^2)$ integrales numéricas
    \item Discretizada: $\mathcal{O}(N)$ integrales locales más simples
\end{itemize}

\textbf{Solución del sistema:}
\begin{itemize}
    \item No discretizada: $\mathcal{O}(M^3)$ con solvers densos
    \item Discretizada: $\mathcal{O}(N)$ con solvers sparse optimizados
\end{itemize}

\subsubsection{Escalabilidad}

\begin{itemize}
    \item \textbf{No discretizada}: Limitada por:
    \begin{itemize}
        \item Ortogonalidad numérica de polinomios de alto orden
        \item Mal condicionamiento de la matriz para $M$ grande
        \item No aplicable a geometrías complejas
    \end{itemize}
    \item \textbf{Discretizada}: Excelente escalabilidad:
    \begin{itemize}
        \item Puede manejar millones de elementos
        \item Aplicable a geometrías arbitrarias 2D/3D
        \item Refinamiento local (h-adaptivity, p-adaptivity)
    \end{itemize}
\end{itemize}

\subsubsection{Memoria}

\begin{itemize}
    \item No discretizada: $\mathcal{O}(M^2)$ almacenamiento
    \item Discretizada: $\mathcal{O}(N)$ almacenamiento con formatos sparse
\end{itemize}

\subsection{¿Con cuál de las dos estrategias se tiene un orden de convergencia mayor?}

\textbf{Respuesta: La formulación NO discretizada con Legendre tiene mayor orden de convergencia.}

\subsubsection{Convergencia Legendre (No Discretizada)}

\begin{itemize}
    \item Tipo: \textbf{Convergencia exponencial (espectral)}
    \item Comportamiento: $\|e\|_{L_2} \sim e^{-\alpha M}$
    \item Ventajas:
    \begin{itemize}
        \item Precisión muy alta con pocos grados de libertad
        \item Ideal para soluciones suaves
        \item $M = 12-15$ suficiente para error $< 10^{-6}$
    \end{itemize}
    \item Desventajas:
    \begin{itemize}
        \item Solo funciona bien si la solución es suave
        \item Limitado a geometrías simples
        \item Convergencia se degrada con discontinuidades (fenómeno de Gibbs)
    \end{itemize}
\end{itemize}

\subsubsection{Convergencia FEM Lagrange (Discretizada)}

\begin{itemize}
    \item Tipo: \textbf{Convergencia algebraica}
    \item Comportamiento: $\|e\|_{L_2} \sim \mathcal{O}(h^{p+1})$ donde $p$ es el orden del elemento
    \item Para elementos lineales: $\mathcal{O}(h^2)$
    \item Ventajas:
    \begin{itemize}
        \item Robusto para todo tipo de problemas
        \item Aplicable a geometrías complejas
        \item Refinamiento local posible
    \end{itemize}
    \item Desventajas:
    \begin{itemize}
        \item Requiere más grados de libertad para igual precisión
        \item $n_{elem} = 50-100$ para error $< 10^{-3}$
    \end{itemize}
\end{itemize}

\subsubsection{Conclusión}

Para este problema 1D con solución suave:
\begin{itemize}
    \item \textbf{Legendre}: Superior en convergencia y eficiencia
    \item \textbf{FEM}: Más práctico para problemas reales complejos
\end{itemize}

\subsection{¿Qué efectos tiene reducir y aumentar $k$?}

\subsubsection{Efectos Físicos}

\textbf{Reducir $k$ (menor tasa de reacción):}
\begin{itemize}
    \item Menor velocidad de consumo de reactivo
    \item Mayor concentración residual a la salida
    \item Menor conversión global del reactor
    \item Perfil de concentración más uniforme
    \item Reactor operando en régimen dominado por advección
\end{itemize}

\textbf{Aumentar $k$ (mayor tasa de reacción):}
\begin{itemize}
    \item Mayor velocidad de consumo de reactivo
    \item Menor concentración a la salida
    \item Mayor conversión global
    \item Perfil de concentración con gradientes pronunciados
    \item Posible formación de capas límite reactivas
    \item Reactor operando en régimen dominado por reacción
\end{itemize}

\subsubsection{Efectos Numéricos}

\textbf{Para $k$ grande:}
\begin{itemize}
    \item Aparecen gradientes fuertes
    \item Mayor dificultad para aproximar numéricamente
    \item Requiere mayor refinamiento (mayor $M$ o menor $h$)
    \item Posible inestabilidad numérica si no se refina adecuadamente
    \item Número de Péclet advectivo-reactivo: $Pe_r = \frac{UL}{\sqrt{kD}}$
    \item Para $k \to \infty$: problema se vuelve \textit{stiff}
\end{itemize}

\textbf{Para $k$ pequeño:}
\begin{itemize}
    \item Solución más suave
    \item Más fácil de aproximar numéricamente
    \item Menor refinamiento necesario
    \item Convergencia más rápida
\end{itemize}

\subsubsection{Relación Conversión vs $k$}

La conversión aumenta con $k$, pero no linealmente:
\begin{equation}
X(k) = f(Da) = f\left(\frac{kL}{U}\right)
\end{equation}

donde $Da$ es el número de Damköhler (relación entre tasa de reacción y tasa de transporte).

\section{Conclusiones}

\begin{enumerate}
    \item Se desarrolló exitosamente la solución del reactor PFR usando dos formulaciones de elementos finitos.

    \item La formulación NO discretizada con Legendre muestra convergencia exponencial, alcanzando alta precisión con pocos grados de libertad ($M \sim 12-15$).

    \item La formulación discretizada con FEM y Lagrange muestra convergencia $\mathcal{O}(h^2)$, requiriendo más elementos ($n_{elem} \sim 50-100$) para precisión comparable.

    \item El método de Galerkin con debilitamiento permite incorporar naturalmente las condiciones de contorno tipo Neumann.

    \item La tasa de reacción $k$ tiene efectos significativos tanto en el comportamiento físico (conversión) como en los requisitos numéricos (refinamiento).

    \item Para problemas 1D con soluciones suaves, los métodos espectrales (Legendre) son superiores en eficiencia computacional.

    \item Para problemas complejos, geometrías irregulares o 2D/3D, los métodos FEM son más prácticos y robustos.

    \item Ambos métodos convergen correctamente a la solución analítica, validando la implementación.
\end{enumerate}

\section{Referencias}

\begin{enumerate}
    \item Reddy, J. N. (2006). \textit{An Introduction to the Finite Element Method}. McGraw-Hill.
    \item Zienkiewicz, O. C., \& Taylor, R. L. (2000). \textit{The Finite Element Method: Volume 1, The Basis}. Butterworth-Heinemann.
    \item Hughes, T. J. R. (2000). \textit{The Finite Element Method: Linear Static and Dynamic Finite Element Analysis}. Dover Publications.
    \item Boyd, J. P. (2001). \textit{Chebyshev and Fourier Spectral Methods}. Dover Publications.
    \item Levenspiel, O. (1999). \textit{Chemical Reaction Engineering}. John Wiley \& Sons.
\end{enumerate}

\end{document}
